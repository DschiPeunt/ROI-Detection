\documentclass[a4paper,12pt]{article}
\usepackage[utf8]{inputenc}
\usepackage[T1]{fontenc}
\usepackage[english]{babel}
\usepackage{lmodern}
\usepackage{amsmath}
\usepackage{amssymb}
\usepackage{amsthm}
\usepackage[superscript]{cite}
\usepackage{nicefrac}
\usepackage{upgreek}
\usepackage{paralist}
\usepackage{stmaryrd}
\usepackage{tikz}
\usepackage{graphicx}
\usepackage{qtree}
\usepackage{dsfont}
\usepackage{eurosym}
\usepackage{setspace}
\usepackage{fancyhdr}
% \usepackage[colorlinks=true,linkcolor=blue]{hyperref}				% Blaue Links sehen meiner Ansicht nach besser aus als die rot umrandeten Verweise

% Kopfzeile
\newcommand\shorttitle{TI - Übungsblatt 01}
\newcommand\authors{Silja Bundesen \& Dominik Blank}

\fancyhf{} % sets all head and foot elements empty
\fancyhead[L]{\shorttitle}
\fancyhead[R]{\authors}
\pagestyle{fancy} % sets the page style to the style delivered and editable with fancyhdr

% Kommandos, Operatoren, etc.
\newcommand{\abs}[1]{\lvert#1\rvert}
\newcommand{\norm}[1]{\lVert#1\rVert}
\DeclareMathOperator{\thetafunc}{\uptheta}

% Mathe-Umgebungen
\theoremstyle{plain}
\newtheorem{theorem}{Satz}
\newtheorem{lemma}{Lemma}
\newtheorem{corollary}{Korollar}
\theoremstyle{definition}
\newtheorem{definition}{Definition}
\theoremstyle{remark}
\newtheorem{remark}{Bemerkung}
\newtheorem{example}{Beispiel}

% ENDE PRÄAMBEL

\author{Dominik Blank}
\title{Gleichmäßige obere Schranken beim Kreisproblem}

\onehalfspacing
\setlength{\parindent}{0pt}
\allowdisplaybreaks

\begin{document}
\begin{titlepage}

\newcommand{\HRule}{\rule{\linewidth}{0.5mm}} % Defines a new command for the horizontal lines, change thickness here

\center % Center everything on the page
 
%----------------------------------------------------------------------------------------
%	HEADING SECTIONS
%----------------------------------------------------------------------------------------

\textsc{\Large Georg-August-Universität Göttingen}\\[1.5cm] % Name of your university/college
\textsc{\large Bachelor-Arbeit zum Thema}\\[0.5cm] % Major heading such as course name
% \textsc{\large Minor Heading}\\[0.5cm] % Minor heading such as course title

%----------------------------------------------------------------------------------------
%	TITLE SECTION
%----------------------------------------------------------------------------------------

\HRule \\[0.4cm]
{\large \bfseries Gleichmäßige obere Schranken beim Kreisproblem}\\[0.2cm] % Title of your document
\HRule \\[1.5cm]
 
%----------------------------------------------------------------------------------------
%	AUTHOR SECTION
%----------------------------------------------------------------------------------------

\begin{minipage}{0.4\textwidth}
\begin{flushleft} \large
\emph{Autor:}\\
Dominik \textsc{Blank} % Your name
\end{flushleft}
\end{minipage}
~
\begin{minipage}{0.5\textwidth}
\begin{flushright} \large
\emph{Betreuer:} \\
Prof. Dr. Valentin \textsc{Blomer} % Supervisor's Name
\end{flushright}
\end{minipage}\\[4cm]

%----------------------------------------------------------------------------------------
%	DATE SECTION
%----------------------------------------------------------------------------------------

{\large Göttingen, den \today}\\[3cm] % Date, change the \today to a set date if you want to be precise

\begin{abstract}
	In dieser Arbeit wird eine asymptotische Formel für die ganzzahligen Darstellungen der natürlichen Zahlen $m \leq R$ durch eine positiv-definite quadratische Form hergeleitet.
	Die implizite Konstante des Fehlerterms wird dabei nicht von der jeweiligen quadratischen Form abhängen.
\end{abstract}
\end{titlepage}

\newpage

\begin{theorem}
	Assume the following statistical model:
	
	Let $M, N \in \mathbb{N}$ and $G = \left\{ 0, \dots, M-1 \right\} \times  \left\{ 0, \dots, N-1 \right\}$. We are given data
	\begin{equation}\label{f}
		F(i, j) = c + V(i, j) + \varepsilon_{i, j}
	\end{equation}
	where $(i, j) \in G$, $c \in \mathbb{R}$ is constant, $V(i, j) \in \{ 0, \pm c \}$ and $\varepsilon_{i, j} \sim \mathcal{N}(0, \sigma^2)$ are i.i.d. normal distributed random variables for some $\sigma > 0$ and for all $(i, j) \in G$.
	
	We assume that $V$ contains a rectangular region of interest. That means, that there are coordinates $(i_{tlc}, j_{tlc}), (i_{brc}, j_{brc}) \in G$ with $i_{tlc} \leq i_{brc}$ and $j_{tlc} \leq j_{brc}$, such that $V(i, j) \neq 0$ if and only if $i_{tlc} \leq i \leq i_{brc}$ and $j_{tlc} \leq j \leq j_{brc}$.
	
	Furthermore assume, that the aforementioned region of interest has a checkerboard pattern, i.e. one of the following relations is true:
	\begin{subequations}
		\begin{align}
			V(i, j) = c &\Leftrightarrow i + j \mathrm{\ is \ odd} \\
			V(i, j) = c &\Leftrightarrow i + j \mathrm{\ is \ even}
		\end{align}
	\end{subequations}
	for all $(i, j) \in R := \{ i_{tlc}, \dots, i_{brc} \} \times \{ j_{tlc}, \dots, j_{brc} \}$.
	
	for all $(i, j)$ with $i < i_{tlc}$ or $j < j_{tlc}$ or $i > i_{brc}$ or $j > j_{brc}$.
	Let $f$ be an image that contains a rectangular ROI. Assume that we are given a binarized image $f_{bin}$ with
	\begin{equation*}
		\mathbb{P}(f_{bin}(i, j) = 1 \mid H_0(i, j)) \leq \alpha
	\end{equation*}
	where $H_0(i, j)$ denotes the null hypothesis for the pixel $(i, j)$, which is, that it is a background pixel and thus should be set to zero.
	
	Let $k \in \mathbb{N}$ be odd and $B$ be a square structuring element with side length $k$. For $\tilde{m}, \tilde{n} \in \{ -\frac{k - 1}{2}, \dots, \frac{k - 1}{2} \}$ we denote by $\mathcal{G}_{(\tilde{m}, \tilde{n})}^k(i, j)$ the set of all possible ground truths in the square with side length $k$, where the pixel $(i, j)$ has offest $(\tilde{m}, \tilde{n})$ from the center of the square and assuming that the null hypothesis for the pixel $(i, j)$ is true.
	Then the following inequality holds:
	\begin{equation*}
		\mathbb{P}((f_{bin} \circ B)(i, j) = 1 \mid H_0(i, j)) \leq k^2 \alpha^k
	\end{equation*}
\end{theorem}
\begin{proof}
	First we notice that for fixed $\tilde{m}, \tilde{n}$ the set $\mathcal{G}_{(\tilde{m}, \tilde{n})}^k(i, j)$ contains ALL possible ground truths given that the null hypothesis for the pixel $(i, j)$ is true, thus
	\begin{equation*}
		\sum_{G \in \mathcal{G}_{(\tilde{m}, \tilde{n})}^k(i, j)} \mathbb{P}(G \mid H_0(i, j)) = 1
	\end{equation*}
	Second we notice, that any element $G \in \mathcal{G}_{(\tilde{m}, \tilde{n})}^k(i, j)$ already contains the null hypothesis for the pixel $(i, j)$, thus giving
	\begin{equation*}
		\mathbb{P}(G \mid H_0(i, j)) = \mathbb{P}(G)
	\end{equation*}
	Third we see that for any possible ground truth $G \in \mathcal{G}_{(\tilde{m}, \tilde{n})}^k(i, j)$ for the whole $k$ by $k$ square to be set to one in $f_{bin}$, there are at least $k$ falsely identified pixels in that square.
	
	Let $K = \{ -\frac{k - 1}{2}, \dots, \frac{k - 1}{2} \}$. Using above observations, we get
	\begin{align*}
		\mathbb{P}&((f_{bin} \circ B)(i, j) = 1 \mid H_0(i, j)) \\
		&= \mathbb{P} \left( \bigcup_{\tilde{m}, \tilde{n} \in K} \bigcap_{m, n \in K} ( f_{bin}(i + m - \tilde{m}, j + n - \tilde{n}) = 1 ) \mid H_0(i, j) \right) \\
		&= \sum_{\tilde{m}, \tilde{n} \in K} \mathbb{P} \left( \bigcap_{m, n \in K} ( f_{bin}(i + m - \tilde{m}, j + n - \tilde{n}) = 1 ) \mid H_0(i, j) \right) \\
		&= \sum_{\tilde{m}, \tilde{n} \in K} \sum_{G \in \mathcal{G}_{(\tilde{m}, \tilde{n})}^k(i, j)} \mathbb{P}(G \mid H_0(i, j)) \cdot \mathbb{P} \left( \bigcap_{m, n \in K} ( f_{bin}(i + m - \tilde{m}, j + n - \tilde{n}) = 1 ) \mid G, H_0(i, j) \right) \\
		&= \sum_{\tilde{m}, \tilde{n} \in K} \sum_{G \in \mathcal{G}_{(\tilde{m}, \tilde{n})}^k(i, j)} \mathbb{P}(G) \cdot \underbrace{\mathbb{P} \left( \bigcap_{m, n \in K} ( f_{bin}(i + m - \tilde{m}, j + n - \tilde{n}) = 1 ) \mid G, H_0(i, j) \right)}_{\leq \alpha^k} \\
		&\leq \sum_{\tilde{m}, \tilde{n} \in K} \sum_{G \in \mathcal{G}_{(\tilde{m}, \tilde{n})}^k(i, j)} \mathbb{P}(G) \cdot \alpha^k \\
		&= \alpha^k \cdot \sum_{\tilde{m}, \tilde{n} \in K} \underbrace{\sum_{G \in \mathcal{G}_{(\tilde{m}, \tilde{n})}^k(i, j)} \mathbb{P}(G)}_{= 1} = \alpha^k \cdot \sum_{\tilde{m}, \tilde{n} \in K} 1 = \alpha^k \cdot \abs{K}^2 = k^2 \alpha^k
	\end{align*}
\end{proof}


\end{document}
