\documentclass[a4paper,12pt]{article}
\usepackage[utf8]{inputenc}
\usepackage[T1]{fontenc}
\usepackage[english]{babel}
\usepackage{lmodern}
\usepackage{amsmath}
\usepackage{amssymb}
\usepackage{amsthm}
\usepackage[superscript]{cite}
\usepackage{nicefrac}
\usepackage{upgreek}
\usepackage{paralist}
\usepackage{stmaryrd}
\usepackage{tikz}
\usepackage{graphicx}
\usepackage{qtree}
\usepackage{dsfont}
\usepackage{eurosym}
\usepackage{setspace}
\usepackage{fancyhdr}
% \usepackage[colorlinks=true,linkcolor=blue]{hyperref}				% Blaue Links sehen meiner Ansicht nach besser aus als die rot umrandeten Verweise

% Kopfzeile
\newcommand\shorttitle{Proofs}
\newcommand\authors{Dominik Blank}

\fancyhf{} % sets all head and foot elements empty
\fancyhead[L]{\shorttitle}
\fancyhead[R]{\authors}
\pagestyle{fancy} % sets the page style to the style delivered and editable with fancyhdr

% Kommandos, Operatoren, etc.
\newcommand{\abs}[1]{\lvert#1\rvert}
\newcommand{\norm}[1]{\lVert#1\rVert}
\DeclareMathOperator{\thetafunc}{\uptheta}

% Mathe-Umgebungen
\theoremstyle{plain}
\newtheorem{theorem}{Theorem}
\newtheorem{lemma}{Lemma}
\newtheorem{corollary}{Corollary}
\theoremstyle{definition}
\newtheorem{definition}{Definition}
\theoremstyle{remark}
\newtheorem{remark}{Remark}
\newtheorem{example}{Example}

% ENDE PRÄAMBEL

\author{Dominik Blank}
\title{Gleichmäßige obere Schranken beim Kreisproblem}

\onehalfspacing
\setlength{\parindent}{0pt}
\allowdisplaybreaks

\begin{document}
%\begin{titlepage}
%
%\newcommand{\HRule}{\rule{\linewidth}{0.5mm}} % Defines a new command for the horizontal lines, change thickness here
%
%\center % Center everything on the page
% 
%%----------------------------------------------------------------------------------------
%%	HEADING SECTIONS
%%----------------------------------------------------------------------------------------
%
%\textsc{\Large Georg-August-Universität Göttingen}\\[1.5cm] % Name of your university/college
%\textsc{\large Bachelor-Arbeit zum Thema}\\[0.5cm] % Major heading such as course name
%% \textsc{\large Minor Heading}\\[0.5cm] % Minor heading such as course title
%
%%----------------------------------------------------------------------------------------
%%	TITLE SECTION
%%----------------------------------------------------------------------------------------
%
%\HRule \\[0.4cm]
%{\large \bfseries Gleichmäßige obere Schranken beim Kreisproblem}\\[0.2cm] % Title of your document
%\HRule \\[1.5cm]
% 
%%----------------------------------------------------------------------------------------
%%	AUTHOR SECTION
%%----------------------------------------------------------------------------------------
%
%\begin{minipage}{0.4\textwidth}
%\begin{flushleft} \large
%\emph{Autor:}\\
%Dominik \textsc{Blank} % Your name
%\end{flushleft}
%\end{minipage}
%~
%\begin{minipage}{0.5\textwidth}
%\begin{flushright} \large
%\emph{Betreuer:} \\
%Prof. Dr. Valentin \textsc{Blomer} % Supervisor's Name
%\end{flushright}
%\end{minipage}\\[4cm]
%
%%----------------------------------------------------------------------------------------
%%	DATE SECTION
%%----------------------------------------------------------------------------------------
%
%{\large Göttingen, den \today}\\[3cm] % Date, change the \today to a set date if you want to be precise
%
%\begin{abstract}
%	In dieser Arbeit wird eine asymptotische Formel für die ganzzahligen Darstellungen der natürlichen Zahlen $m \leq R$ durch eine positiv-definite quadratische Form hergeleitet.
%	Die implizite Konstante des Fehlerterms wird dabei nicht von der jeweiligen quadratischen Form abhängen.
%\end{abstract}
%\end{titlepage}

\newpage

\begin{definition}
	Let $M, N \in \mathbb{N}$ and $V \in \mathbb{R}^{M \times N}$ be a matrix. Assume there are two pairs of indices $(i_{tlc}, j_{tlc}), (i_{brc}, j_{brc})$ with $1 \leq i_{tlc} \leq i_{brc} \leq M$ and $1 \leq j_{tlc} \leq j_{brc} \leq N$, such that
	\begin{equation}
		V(i, j) \neq 0 \textrm{ if and only if } (i, j) \in \{ i_{tlc}, \dots, i_{brc} \} \times \{ j_{tlc}, \dots, j_{brc} \}
	\end{equation}
	We call $R = \{ i_{tlc}, \dots, i_{brc} \} \times \{ j_{tlc}, \dots, j_{brc} \}$ a \emph{rectangular region of interest (rROI)} and say that $V$ contains the rROI $R$.
	
	Furthermore, we call $(i_{tlc}, j_{tlc})$ the \emph{top left corner} and $(i_{brc}, j_{brc})$ the \emph{bottom right corner} of the rROI.
\end{definition}

\begin{definition}
	Let $M, N \in \mathbb{N}$ and $c \in \mathbb{R} \setminus \{ 0 \}$. Let $V \in \mathbb{R}^{M \times N}$ be a matrix, that only takes values in the set $\{ 0, \pm c \}$ and that contains a rectangular region of interest $R$. We say that $R$ has a \emph{checkerboard pattern}, if one of the following relations is true:
	\begin{subequations}
		\begin{align}
			\textrm{For all } (i, j) \in R: V(i, j) = c &\textrm{ if and only if } i + j \textrm{ is odd}.\\
			\textrm{For all } (i, j) \in R: V(i, j) = c &\textrm{ if and only if } i + j \textrm{ is even}.
		\end{align}
	\end{subequations}
\end{definition}

\begin{remark}
	If the first relation in the definition is true, that immediately implies, that
	\begin{equation*}
		\textrm{for all } (i, j) \in R: V(i, j) = - c \textrm{ if and only if } i + j \textrm{ is even},
	\end{equation*}
	since $V$ takes only values in $\{ 0, \pm c \}$, but for $(i, j) \in R$ we have $V(i,j) \neq 0$.
	Similarly, if the second relation is true, it implies that
	\begin{equation*}
		\textrm{for all } (i, j) \in R: V(i, j) = - c \textrm{ if and only if } i + j \textrm{ is odd}.
	\end{equation*}
	In both cases the values of $V$ alternate between $+c$ and $-c$ along the rows and columns of $R$. This is similar to the classical checkerboard pattern.
\end{remark}

\newpage

\begin{theorem}
	Assume the following statistical model:
	
	Let $M, N \in \mathbb{N}$ and $G = \left\{ 1, \dots, M \right\} \times  \left\{ 1, \dots, N \right\}$. We are given data
	\begin{equation}\label{f}
		F(i, j) = c + V(i, j) + \varepsilon_{i, j}
	\end{equation}
	where $(i, j) \in G$, $c \in \mathbb{R}$ is constant, $V(i, j) \in \{ 0, \pm c \}$ and $\varepsilon_{i, j} \sim \mathcal{N}(0, \sigma^2)$ are i.i.d. normal distributed random variables for some $\sigma > 0$ and for all $(i, j) \in G$.
	
	Assume that $V$ contains a rectangular region of interest $R$ and that $R$ has a checkerboard pattern.
	
	Let $\varphi_\alpha \in \{ 0, 1 \}^{M \times N}$ be the binary matrix, that represents the decision of the testing procedure to a given statistical significance $\alpha$.
	
	Let $k \in \mathbb{N}$ be odd and $B$ be a square structuring element with side length $k$.
	
	Then the following inequality holds:
	\begin{equation*}
		\mathbb{P}((\varphi_\alpha \circ B)(i, j) = 1 \mid H_0(i, j)) \leq k \alpha^{\frac{k+1}{2}}
	\end{equation*}
\end{theorem}
\begin{proof}
	We aim to find an upper bound for the probability
	\begin{equation*}
		\mathbb{P}((\varphi_\alpha \circ B)(i, j) = 1 \mid H_0(i, j))
	\end{equation*}
	To do this, we first notice that $H_0(i, j)$ is equivalent to $V(i, j) = 0$, but since $V$ contains a rectangular region of interest, this means that $i < i_{tlc}$ or $i > i_{brc}$ or $j < j_{tlc}$ or $j > j_{brc}$. We need to differentiate cases here. These four cases are not mutually exclusive, but have different implications for the row/column of the index $(i, j)$ and a neighbouring row/column:\\
	
	\begin{tabular}{|c|c|c|}
		\hline
		case & row/column of $(i, j)$ & neighbouring row/column \\
		\hline
		$i < i_{tlc}$ & $V(i, 1) = \dots = V(i, N) = 0$ & $V(i-1, 1) = \dots = V(i-1, N) = 0$ \\
		\hline
		$i > i_{brc}$ & $V(i, 1) = \dots = V(i, N) = 0$ & $V(i+1, 1) = \dots = V(i+1, N) = 0$ \\
		\hline
		$j < j_{tlc}$ & $V(1, j) = \dots = V(M, j) = 0$ & $V(1, j-1) = \dots = V(M, j-1) = 0$ \\
		\hline
		$j > j_{brc}$ & $V(1, j) = \dots = V(M, j) = 0$ & $V(1, j+1) = \dots = V(M, j+1) = 0$ \\
		\hline
	\end{tabular}\\
	
	Without loss of generality we assume the first case. This means that the null hypotheses $H(i, 1), \dots, H(i, N)$ and $H(i-1, 1), \dots, H(i-1, N)$ are true.
	
	To be even more precise, it implies that $D^-(i, 1) = \dots = D^-(i, N) = 0$.
	
	We have assumed the side length $k$ of the structuring element $B$ to be odd.
	
	We define the two index sets $K = \{ -\frac{k - 1}{2}, -\frac{k - 3}{2}, \dots, \frac{k - 3}{2}, \frac{k - 1}{2} \}$ and $K_1 = \{ -\frac{k - 1}{2}, -\frac{k - 5}{2}, \dots, \frac{k - 5}{2}, \frac{k - 1}{2} \}$. This yields
	\begin{align*}
		\mathbb{P}&((\varphi_\alpha \circ B)(i, j) = 1 \mid H_0(i, j)) \\
		&= \mathbb{P} \left( \bigcup_{\tilde{m}, \tilde{n} \in K} \bigcap_{m, n \in K} \{ \varphi_\alpha(i + m - \tilde{m}, j + n - \tilde{n}) = 1 \} \mid D^-(i, 1) = \dots = D^-(i, N) = 0 \right) \\
		&= \sum_{\tilde{n} \in K} \mathbb{P} \left( \bigcup_{\tilde{m} \in K} \bigcap_{m, n \in K} \{ \varphi_\alpha(i + m - \tilde{m}, j + n - \tilde{n}) = 1 \} \mid D^-(i, 1) = \dots = D^-(i, N) = 0 \right) \\
		&\leq \sum_{\tilde{n} \in K} \mathbb{P} \left( \bigcup_{\tilde{m} \in K} \bigcap_{n \in K, m = \tilde{m}} \{ \varphi_\alpha(i + m - \tilde{m}, j + n - \tilde{n}) = 1 \} \mid D^-(i, 1) = \dots = D^-(i, N) = 0 \right) \\
		&= \sum_{\tilde{n} \in K} \mathbb{P} \left( \bigcup_{\tilde{m} \in K} \bigcap_{n \in K} \{ \varphi_\alpha(i, j + n - \tilde{n}) = 1 \} \mid D^-(i, 1) = \dots = D^-(i, N) = 0 \right) \\
		&= \sum_{\tilde{n} \in K} \mathbb{P} \left( \bigcap_{n \in K} \{ \varphi_\alpha(i, j + n - \tilde{n}) = 1 \} \mid D^-(i, 1) = \dots = D^-(i, N) = 0 \right) \\
		&\leq \sum_{\tilde{n} \in K} \mathbb{P} \left( \bigcap_{n \in K_1} \{ \varphi_\alpha(i, j + n - \tilde{n}) = 1 \} \mid D^-(i, 1) = \dots = D^-(i, N) = 0 \right) \\
		&\leq \sum_{\tilde{n} \in K} \prod_{n \in K_1} \mathbb{P} \left( \{ \varphi_\alpha(i, j + n - \tilde{n}) = 1 \} \mid D^-(i, 1) = \dots = D^-(i, N) = 0 \right) \\
		&= \sum_{\tilde{n} \in K} \prod_{n \in K_1} \mathbb{P} \left( T(i, j + n - \tilde{n}) \geq t_\alpha \mid D^-(i, 1) = \dots = D^-(i, N) = 0 \right) \\
		&\leq \sum_{\tilde{n} \in K} \prod_{n \in K_1} \mathbb{P} \left( \tilde{D}^-(i, j + n - \tilde{n}) \geq t_\alpha \mid D^-(i, 1) = \dots = D^-(i, N) = 0 \right) \\
		&= \sum_{\tilde{n} \in K} \prod_{n \in K_1} \mathbb{P} \left( \tilde{D}^-(i, j + n - \tilde{n}) \geq t_\alpha \mid D^-(i, j + n - \tilde{n}) = 0 \right) \\
		&\leq \sum_{\tilde{n} \in K} \prod_{n \in K_1} \alpha \\
		&= \abs{K} \alpha^{\abs{K_1}} \\
		&= k \alpha^{\frac{k+1}{2}}
	\end{align*}
	
	The other three cases can be proven in a similar way by swapping the roles of $m$ or $\tilde{m}$ with $n$ or $\tilde{n}$, respectively and/or by replacing $D^-$ by $D^+$.
\end{proof}


\end{document}
